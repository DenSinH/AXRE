\chapter{UCode}

The main interesting part of the DSP's workings is the actual UCode itself. The main entrypoint (for Mortal Kombat 5 at least), is at \hex{10}. The main thing it does is waiting for mail, and then processing a stream of commands (at \hex{0xc00} in DMEM).

The start of the UCode looks like this:

\inputminted[fontsize=\small]{asm}{../ucode/main_loop.asm}

The \mintinline{asm}{command_jump_table} is a table with commands \hex{0} through \hex{11}, though the bounds check also allows for a command \hex{12} to exist. 

Pseudocode for this part could be 

\inputminted{C}{../ucode/main_loop.c}

\section{Commands}
The commands all return with a \mintinline{asm}{JMP receive_command}, save for command \hex{f}, which does some sort of reset.

\subsection{Command \hex{0}}
The assembly looks like 

\inputminted[fontsize=\small]{asm}{../ucode/command_0.asm}

The point of this is to fill 3 regions of memory with either 0's, or incrementing values. Which of the 2 depends on the values from a \hex{40} byte stream DMAd from main memory. 

Note that we are reading a \mintinline{c}{base} and an \mintinline{c}{incr} 9 times from the stream, which would amount to 9 * \hex{6} = \hex{0x36} bytes, so the DMA transfers 4 bytes too many.

I suspect that the incrementing values are a main memory address and strides. The address regions \hex{0000} - \hex{03c0}, \hex{0400} - \hex{07c0} and \hex{07c0} - \hex{0b80} will be used in most other commands.

Pseudocode for this could be 

\inputminted{cpp}{../ucode/command_0.c}

\subsection{Command \hex{1}}
\subsection{Command \hex{2}}
\subsection{Command \hex{3}}
\subsection{Command \hex{4}, \hex{5} and \hex{9}}
These commands are all very similar. Command \hex{9} only calls \mintinline{c}{sub_484} with a pointer to the buffer at \hex{7c0}, while \hex{4} and \hex{5} DMA the buffers at \hex{400} and \hex{7c0} respectively, before also calling \mintinline{c}{sub_484} with their respective buffers as arguments. Since they are so similar, I will only put the assembly for command \hex{4} here.

\inputminted{asm}{../ucode/command_4.asm}

And the pseudocode for \hex{4} and \hex{5} is the same, except \hex{5} uses \hex{7c0} instead of \hex{400}:

\inputminted{c}{../ucode/command_459.c}

\subsection{Command \hex{6}}
\subsection{Command \hex{7}}
\subsection{Command \hex{8}}
\subsection{Command \hex{a} - \hex{c}}
These commands immediately return on call.
\subsection{Command \hex{d}}
\subsection{Command \hex{e}}
\subsection{Command \hex{f}}
\subsection{Command \hex{10}}
\subsection{Command \hex{11}}