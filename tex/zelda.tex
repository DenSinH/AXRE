\chapter{Zelda}

The Zelda UCode is another popular UCode. The main entry for this ucode has the following assembly:

\inputminted[fontsize=\small]{asm}{../zelda/main_loop.asm}

The entry point for this ucode is either \hex{0} or \hex{10}. Because of the way the switch case looks, it is hard to specifically say how many commands there really are. The main loop pseudocode looks like 

\inputminted[]{c}{../zelda/main_loop.c}

Every command seems to end by jumping to the implementation of command \hex{0}, which sends the CPU mail. The assembly for this looks like 

\inputminted[fontsize=\small]{asm}{../zelda/command_0.asm}

And pseudocode looks like

\inputminted[]{c}{../zelda/command_0.c}

From now, I will just disassemble and implement the commands as I come across them, simply because of the strange structure of them.


\section{DMA Functions}
Some DMA functions used in several places:

\inputminted[fontsize=\small]{asm}{../zelda/dma_funcs.asm}

with pseudocode

\inputminted[fontsize=\small]{asm}{../zelda/dma_funcs.c}

\section{Command \hex{1}}

Assembly:

\inputminted[fontsize=\small]{asm}{../zelda/command_1.asm}

with pseudocode

\inputminted[fontsize=\small]{asm}{../zelda/command_1.c}